\documentclass[10pt]{article}
\usepackage{geometry}
\setlength{\oddsidemargin}{0in}
\setlength{\evensidemargin}{0in} \setlength{\textwidth}{6.4in}
\geometry{letterpaper, top=1.5cm, bottom=1.5cm, left=2cm}
\usepackage[colorlinks=true, urlcolor=urlblue]{hyper ref}

\usepackage{color}
\definecolor{urlblue}{rgb}{0, 0, 0.5} % less intense blue

\renewcommand{\familydefault}{cmss}


\newenvironment{packed_list}{
\begin{itemize}
  \setlength{\itemsep}{1pt}
  \setlength{\parskip}{0pt}
  \setlength{\parsep}{0pt}
}{\end{itemize}}


\begin{document}
\centerline{\bf \large BIOSTATS 743 (3 credits)} 
\centerline{\bf \large Categorical Data Analysis}
\centerline{\bf Fall 2018 :: T/Th 1:00-2:15  :: LGRT 141 }

\vspace{.25in}
\noindent {\sc instructor}\\
\noindent Nicholas G Reich \\
\noindent 425 Arnold House \\
\noindent (413) 545-4534 \\
\noindent nick [at] umass.edu \\
\noindent twitter: \href{https://twitter.com/reichlab}{@reichlab}\\
\noindent web: \href{https://reichlab.io}{reichlab.io}\\
\noindent Office Hours: Tuesday 2:30-3:30pm \\
\noindent Teaching assistant: Zhengfan Wang \\
%\noindent \href{https://github.com/nickreich/applied-regression-2016}{Course notes on GitHub}\\
%\noindent \href{https://piazza.com/umass/spring2016/pubhlth690nr/home}{Piazza site}
%\noindent \href{https://umass.echo360.com/ess/portal/section/6c47935b-4969-45f0-8498-904023f6eb3f}{Video lectures}

%\bigskip
%\noindent {\sc Teaching assistant}:
%TBD

\bigskip
\noindent {\sc Materials}

\noindent {\em Required Textbooks}

Categorical Data Analysis, Author: Alan Agresti, Publisher: Wiley, 3rd Edition.

\noindent {\em Recommended Textbooks}

Andrew Gelman and Jennifer Hill. (2007). \emph{Data Analysis using Regression and Multilevel/Hierarchical Models}. Cambridge University Press.
%Hosmer DW and Lemeshow S. (2000). \emph{Applied Logistic Regression}. \emph{2nd Edition}. %Wiley.

% Introduction to Statistical Thought by Michael Lavine, Chair of UMass Math/Stat \href{http://www.math.umass.edu/~lavine/Book/book.html}{[free PDF]}

%\bigskip
%\noindent {\sc Software} (both free downloads):\\

%RStudio :: \href{http://www.rstudio.org}{rstudio.org}


\bigskip
\noindent {\sc Prerequisites}\\
BIOSTATS 540 (Intro Biostatistics), STAT 515 (Intro Statistics I), STAT 516 (Intro Statistics II), BIOSTATS 650/STAT 525 (both of these are courses on linear regression), or equivalent coursework. Prior programming experience is required. If you have not taken some of the listed pre-requisites but still think the course is the right choice for you, you may petition the instructor directly for permission to enroll in the course.


%\bigskip
%\noindent {\sc Optional recommended readings:}\\
%\noindent Visualize This: the FlowingData guide to design, visualization and statistics by Nathan Yau, Wiley (2011)\\
%\noindent Any of Edward Tufte's four big books \\
%\noindent Information is Beautiful, David McCandless \\
%\noindent FlowingData blog \@ flowingdata.com 


\bigskip
\noindent {\sc Course Description}\\
This course provides an overview of statistical methods for analyzing data where the outcome variable is categorical or discrete. The course will emphasize the theoretical underpinnings of the methods as well as an applied understanding of the computation and interpretation, both of which are necessary to succeed with real data analysis. We will cover inference for binomial and multinomial variables with contingency tables, generalized linear models, logistic regression for binary responses, logit models for multiple response categories, log-linear models, some machine  learning approaches, inference for matched-pairs, and correlated/clustered data. Examples will be taken from public health and biomedical research. Students will be evaluated on homework assignments, a mid-term exam and a final project.
 
%The aim of this course is to provide fundamental statistical concepts and tools relevant to the practice of summarizing, analyzing, and visualizing data. This course will build your knowledge of the fundamental principles of biostatistical inference.  We will focus on linear regression and generalized linear regression models using a variety of examples and exercises from medical and public health research. %{\bf I hope that you come out of this class with a more complete and nuanced understanding of the practice and theory of statistics -- the kind of understanding that can only be gained by rolling up your sleeves and working with real data.}  

%Distribution and inference for binomial and multinomial variables with contingency tables, generalized linear models, logistic regression for binary responses, logit models for multiple response categories, loglinear models, inference for matched-pairs and correlated clustered data. Textbook chapters 1-14 are to be covered if time allows.

\bigskip
\noindent {\sc Software}\\
The textbook website \href{http://www.stat.ufl.edu/~aa/cda/cda.html} lists the datasets used for examples, solutions to some exercises, information about using R, SAS, Stata, and SPSS software for conducting the analyses in the text. I will use R throughout the course. Students are allowed to use the language of their choice for all assignments and projects. The instructor will only be able to provide support for assignments done in R and solutions will be given in terms of R only. R is freely available at \href{http://cran.r-project.org/}. R is also installed at every OIT classroom. See \href{http://www.umass.edu/statdata/} for more information. 
 
%\bigskip
%\noindent {\sc Course Goals}\\
%At the end of the class, successful students will be familiar with various estimation and inference tools for categorical data and will be able to apply appropriate methods to specific settings. The students will have workable R knowledge for categorical data analysis, as well as good understanding of theory and derivation behind scenes.


%\bigskip
%\noindent {\sc Learning Goals} {\em (By the end of the course students will be able to...)}
%
%\begin{itemize}
%\item perform advanced programming tasks in R,
%\item interpret regression output in a scientific context,
%%\item reason quantitatively in the presence of probabilistic uncertainty,
%\item independently formulate, fit, and interpret regression models to weigh evidence for/against hypotheses about associations between variables, 
%\item diagnose the ``goodness-of-fit'' of a given regression model, both on its own and relative to other regression models,
%\item create powerful data visualizations (using R package {\tt ggplot2}) that reveal features of data or fitted models,
%\item design and run simulation studies, 
%\item write concise, professional reproducible statistical analysis reports using {\tt knitr} and RMarkdown.
%%\item write succinct and accurate summaries of data analyses and computational algorithms, and
%%\item design and create a dynamic and visually arresting scientific poster. 
%\end{itemize}
    
% \clearpage
\bigskip
\noindent {\sc Expectations} 

%This course will require you to work thoughtfully, carefully, and independently and will require substantial work outside of class time. Because we will be using a more project-driven approach in this course, with assignments that will build upon one another into a final product, it is vital that you do not fall behind. If you feel as though you are falling behind or starting to lose a handle on the content, I expect you to come talk to me either after class or during office hours so that I can help as much as I can to set you back on track. Please do not wait to talk to me if you start to fall behind.\\
 
%Additionally, this is a new course, that I will be developing and tweaking as we go through the semester. I would be very interested to hear your thoughts, constructive criticism and praise about the activities and content of the course. Please let me know (as we go or at the end) what is and is not working for you.

\noindent Things you should expect from me:
\vskip-5em
\begin{packed_list}
\vskip5em
\item response to questions via Piazza in $<2$ working days (often sooner)
\item attention to your questions related to coursework during office hours
\item explanations about the statistical methods we cover in class
\item instruction in how to write, research, and debug R code
\end{packed_list}

\noindent Things you should not expect from me:
\vskip-5em
\begin{packed_list}
\vskip5em
\item time for frequent non-office hour drop-in questions
\item comments on a research project that is unrelated to your coursework
\item writing your code for you or {\em extensive} debugging of your code
%\item a week-by-week breakdown of in-class activities and topics (the course is in development!)
\end{packed_list}

\bigskip

\clearpage
\noindent {\sc Types of Assignments and Activities, with Grade Contributions}

\noindent Homework (20\%): You will have homework assignments due about every other week throughout the semester. Some parts of the assignments will require you to submit a digital file with reproducible solutions, e.g. a knitr file that reproduces your answers. The homeworks will graded on a discrete scale: Complete (10 points), Submitted but incomplete/inadequate (5 points), Not submitted (0 points). Convincing and thorough attempts at all assigned homework problems will qualify as a ``complete'' submission. Late homeworks will not be accepted under any circumstances. If a homework is not handed in on time, it will receive a grade of zero.\\

\noindent Midterm exam (25\%): The exam will be in-class and closed book. Calculators are not allowed. Make-up exams will only be given for legitimate, documented reasons and when a call has been made to the instructor or the department before the exam. \\

\noindent Scribing (15\%): Twice throughout the semester, each student will be required to transcribe the notes from the white-board in class into an RMarkdown file. Additionally, each student will serve twice as a peer-reviewer for the notes scribed by another student. Each set of transcribed notes will be evaluated for accuracy, completeness, formatting, and readability. Students are encouraged to add or embellish the content presented in class with additional examples or explanations that they think others would find helpful in understanding the material. The notes must follow the format used in the notes lectures 1 and 2. (Templates available on the GitHub course repository.) Final draft of notes are due at 5pm one week after the class during which they were transcribed. Notes must be submitted via pull-request to the course GitHub repository. \\

\noindent Participation/citizenship (10\%) :
Being a good class ``citizen'' plays a large role in your final grade. A few of the characteristics of good class citizens are: attending all course meetings, using office hours, asking questions, offering to answer questions, actively listening when others are talking, correcting typos or errors in course materials on GitHub, and participating on Slack (both asking and answering questions). Citizenship is more a function of quality than quantity. The ``default'' citizenship score is 5 out of 10. [Acknowledgments to Aaron Swoboda for introducing me to the concept of course citizenship and for some of this text.] \\
%from Aaron Swoboda: I consider course citizenship to be a vital part of your grade. A few of the characteristics of good class citizens are: attending all course meetings, using office hours, asking questions, offering to answer questions, actively listening when others are talking, and posting to online discussion forums, among others. Citizenship is more a function of quality than quantity. Note that the "default" citizenship score is 5 out of 10, which allows students who actively and productively contribute to class to substantially increase their grade. Please note that good citizenship is different from "talking a lot," and it is quite possible to earn a low citizenship score because you fail to let others contribute.


\noindent Final Project (30\%): A large component of the course will be an independent project which will be presented to your classmates. A separate handout will provide details. \\


\noindent Extra Credit: If you send me an email with ``I read the syllabus'' as the subject line by the beginning of the second class, you will receive one point of extra credit on your final grade. 

\bigskip
\noindent {\sc Course Policies}

Collaboration on homework is expected and encouraged, although you must write up your own assignment. No copying or cutting and pasting. The exam must be completed without assistance from your classmates. Your independent projects must be your own work. You may discuss your project with others and even solicit ideas and advice, but at the end of the day, you must complete all the analysis and write-up on your own. Any explicitly borrowed ideas must be cited appropriately.

Late assignments: Completing homework assignments on time will be vital to not falling behind in this course. It is expected that you hand in assignments on time. If an assignment is handed in late, you will receive zero credit for that assignment. I will drop your lowest homework grade at the end of the semester.

Make-up exam: Make-up exams will only be given for legitimate, documented reasons and when a call has been made to the instructor or the department before the exam.

Attendance is required. Absences (excused or not) will impact your participation grade.

All mobile devices that can/will be distracting to you or others during class must be turned off at the start of class and may not be used during class time.
  
  \clearpage
  
  {
\bigskip
\noindent {\sc Course Schedule}
  \begin{table}[htp]
\begin{tabular}{ll}
Week & Topic \\
1 & Introduction: Distributions and Inference for Categorical Data \\
2 & Contingency Tables \\
3 & Introduction to Generalized Linear Models \\
4-5 & Logistic Regression \\
6 & Alternative Modeling of Binary Response Data \\
7 & Models for Multinomial Responses \\
8-9 & Loglinear Models for Count Data \\
10 & Models for Matched Pairs \\
11-12 & Clustered Categorical Data: Random Effects Models \\
13 & Other Mixture Models for Discrete Data
\end{tabular}
\end{table}%
}

{ 
\bigskip
\noindent {\sc Grading Scale}
\begin{table}[htp]
\begin{tabular}{ll}
Grade & Percentage \\
\hline
A & 93-100 \\
A- & 90-92 \\
B+ & 87-89 \\
B & 83-86 \\
B- & 80-82 \\
C+ & 77-79 \\
C & 70-76 \\
F & 0-69 \\
\end{tabular}
\end{table}%
  }
  
%  \clearpage
%\bigskip
%\noindent {\sc Formal CEPH Course Competencies}
%\begin{itemize}
%\item Describe the role biostatistics serves in public health.
%\item Distinguish among the different measurement scales and the implications for selection of statistical methods to be used based on these distinctions.
%\item Describe conceptual frameworks (statistical literacy) in biostatistics
%\item Apply biostatistical methods to the design of studies in public health.
%\item Use computers to appropriately store, manage, manipulate and process data for a research study using modern software.
%\item Apply descriptive techniques commonly used to summarize public health data. 
%\item Describe the basic concepts of probability, random variation and selected, commonly used, probability distributions.
%\item Select and perform the appropriate descriptive and inferential statistical methods in selected basic study design settings.
%\item Describe appropriate methodological alternatives to commonly used statistical methods when assumptions are violated.
%\item Integrate analysis strategies in biostatistics with principles and issues in epidemiology.
%\item Apply basic informatics techniques with vital statistics and public health records in the description of public health characteristics.
%\item Interpret results and critically evaluate basic statistical aspects of public health research and practice reported in the literature.
%\item Assist in the application of statistical theory to applied statistical problems.
%\item Develop a conceptual framework that integrates techniques and methods in biostatistics 
%\item Critically evaluate statistical aspects of public health research reported in the literature
%\item Develop written and oral presentations based on statistical analyses for both public health professionals and educated lay audiences.
%\item Apply statistical methods to solve problems in the health sciences and carry out theoretical research in statistical methodology.
%\end{itemize}
  
{\footnotesize 
  
\bigskip
\noindent {\sc Academic Honesty Policy Statement}
Since the integrity of the academic enterprise of any institution of higher education requires honesty in scholarship and research, academic honesty is required of all students at the University of Massachusetts Amherst.

Academic dishonesty is prohibited in all programs of the University. Academic dishonesty includes but is not limited to: cheating, fabrication, plagiarism, and facilitating dishonesty. Appropriate sanctions may be imposed on any student who has committed an act of academic dishonesty. Instructors should take reasonable steps to address academic misconduct. Any person who has reason to believe that a student has committed academic dishonesty should bring such information to the attention of the appropriate course instructor as soon as possible. Instances of academic dishonesty not related to a specific course should be brought to the attention of the appropriate department Head or Chair. The procedures outlined below are intended to provide an efficient and orderly process by which action may be taken if it appears that academic dishonesty has occurred and by which students may appeal such actions.

Since students are expected to be familiar with this policy and the commonly accepted standards of academic integrity, ignorance of such standards is not normally sufficient evidence of lack of intent.
For more information about what constitutes academic dishonesty, please see the \href{http://umass.edu/dean_students/codeofconduct/acadhonesty/}{Dean of Students? website}.

}

{\footnotesize 
\bigskip
\noindent {\sc Disability Statement}
The University of Massachusetts Amherst is committed to making reasonable, effective and appropriate accommodations to meet the needs of students with disabilities and help create a barrier-free campus. If you are in need of accommodation for a documented disability, register with Disability Services to have an accommodation letter sent to your faculty. It is your responsibility to initiate these services and to communicate with faculty ahead of time to manage accommodations in a timely manner. For more information, consult the \href{http://www.umass.edu/disability/}{Disability Services website}.
}

\end{document}

